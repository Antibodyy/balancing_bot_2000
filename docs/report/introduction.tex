\section*{Motivation}
Two-wheeled inverted pendulum (TWIP) robots combine an unstable nonlinear plant with
strict actuator limits, making them ideal benchmarks for advanced control.
Classical LQR control stabilizes the upright equilibrium but struggles once slopes,
trajectory tracking, or torque saturation enter the picture. Our project revisits
this platform with modern tooling: a MuJoCo-based digital twin, CasADi-powered
model predictive control (MPC), and a hardware stack built on the Pololu Balboa
32U4 and Raspberry Pi 4.

\section*{Project Goals}
We established four goals at the outset:
\begin{enumerate}[leftmargin=*]
    \item Develop a validated sixth-order dynamics model that captures slope
          effects and actuator torque limits while remaining tractable for
          real-time MPC.
    \item Implement an MPC controller that runs at 16~Hz, enforces constraints,
          and tracks velocity/heading references in simulation.
    \item Compare MPC performance against an LQR baseline across balancing,
          drive-stop, and trajectory-following scenarios to quantify the benefit
          of constraint handling.
    \item Exercise the same control stack on hardware, including the safety
          interlocks and serial interface required for laboratory testing.
\end{enumerate}

\section*{Completed Work}
The current report documents the pieces we have finished so far:
\begin{itemize}[leftmargin=*]
    \item A simplified yet slope-aware nonlinear model derived with CasADi that
          omits high-frequency wheel gyroscopic coupling for computational
          efficiency, along with analytical linearization and discretization
          routines.
    \item A linear MPC implementation with configurable horizons, warm-started
          QP solves (8--12~ms typical), and comprehensive constraint coverage
          for pitch, pitch rate, and motor torques.
    \item Regression-tested MuJoCo simulations covering perturbation recovery,
          circular motion, and yaw-rate limitations, plus a comparative LQR
          controller using identical plant assumptions.
    \item A hardware integration path that reuses the simulation controller,
          adds timing telemetry, and communicates with the Balboa microcontroller
          through a documented serial protocol.
\end{itemize}

\section*{Paper Outline}
The remainder of the paper proceeds as follows: Section~II details the dynamics
model and linearization/discretization pipeline. Section~III introduces the MPC
formulation, constraints, and warm-start strategy. Section~IV discusses design
parameter tuning while Section~V contrasts MPC with the implemented LQR
baseline. Section~VI presents current conclusions, Section~VII reflects on open
issues and lessons learned, and the Appendix summarizes hardware specifications
alongside implementation artifacts.
