\section{Conclusion}
This project evaluated the practicality of running a constrained linear MPC stack on a low-cost two-wheeled inverted pendulum platform. We derived a slope-aware sixth-order model, implemented online linearization and zero-order-hold discretization, and exercised the controller both in MuJoCo and on the Balboa hardware with a shared software stack. The experiments demonstrate that MPC regulates pitch effectively across disturbances, manages actuator saturation during drive-stop maneuvers, and outperforms a tuned LQR baseline whenever torque bounds are active.

At the same time, the BALANCE regression reveals a structural limitation of linear MPC: with an initial $3^\circ$ pitch perturbation the solver remains optimal yet allows forward velocity drift once the system departs too far from the linearization point. Increasing the horizon, tightening terminal constraints, and boosting terminal costs improved recovery margins but did not eliminate the drift, highlighting the need for richer models or hybrid controllers when operating near the edge of the feasible set.

Future work will extend the controller to nonlinear MPC or capture-point constraints, incorporate improved yaw modeling to close the loop on circular trajectories, and perform more extensive hardware testing to quantify real-world robustness at the 16~Hz update rate. Despite the outstanding limitations, the current system provides a reproducible baseline for balancing research and a foundation for exploring advanced control strategies on resource-constrained platforms.
