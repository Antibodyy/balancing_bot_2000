\subsection*{Cost Function}
The MPC uses a quadratic cost function with separate state and control penalties:

\subsubsection*{Stage Cost}
\[
J = \sum_{k=0}^{N-1} \left[ (x_k - x_{ref,k})^T Q (x_k - x_{ref,k}) + u_k^T R u_k \right]
\]

\subsubsection*{Terminal Cost}
\[
J_N = (x_N - x_{ref,N})^T P (x_N - x_{ref,N})
\]

\subsubsection*{Cost Matrices}

\begin{itemize}[leftmargin=*]
    \item \textbf{$Q$ (State Cost Matrix)}: $6 \times 6$ diagonal matrix
    \begin{itemize}
        \item Penalizes deviations in the state vector
    \end{itemize}

    \item \textbf{$R$ (Control Cost Matrix)}: $2 \times 2$ diagonal matrix
    \begin{itemize}
        \item Penalizes control effort for the input
    \end{itemize}

    \item \textbf{$P$ (Terminal Cost)}: Computed via Discrete Algebraic Riccati Equation (DARE)
\end{itemize}

The cost function prioritizes balance stabilization (pitch angle), smooth control inputs, and trajectory tracking for position and heading.
\vspace{0.5\baselineskip}
\subsubsection*{Initial State}

The initial state $x_0 \in \mathbb{R}^6$ is provided at runtime from sensor measurements and fed back into the MPC model.

\subsection*{Constraints}
There is a state constraint on the pitch of \hl{measure}, measured to ensure the robot will at no point lean on the safety rails. The pitch rate constraint \hl{from code} is added to ensure feasibility with any terminal constraint. A safety constraint is also put in place: when pitch reaches 90 or above, meaning the robot has definitely fallen over beyond recovery, all control actions are stopped.

Furthermore there is symmetric torque limits for both motors:
\begin{align*}
|\tau_i| &\leq \tau_{\max} \quad \text{for } i \in \{1,2\}
\end{align*}

This ensures actuator saturation is respected and the hardware limitations are respected (motors stall torque $\approx 0.25$ [Nm]).

\subsection*{Horizon Length}

The prediction horizon N must be long enough to capture pitch stabilization dynamics while keeping computational cost achievable by using a realistic sampling period $T_s$. The total horizon duration is defined as $N \times T_s$ seconds.

\subsection*{Complete Receding Horizon Problem}

The MPC solves the following optimization problem at each time step:

{\small
\begin{subequations}
\begin{align}
\min_{u_0, \ldots, u_{N-1}} \quad & \sum_{k=0}^{N-1} \big[ (x_k - x_{ref,k})^T Q (x_k - x_{ref,k}) \notag \\
& \quad + u_k^T R u_k \big] + (x_N - x_{ref,N})^T P (x_N - x_{ref,N}) \label{eq:cost} \\
\text{s.t.} \quad & x_{k+1} = A x_k + B u_k, \quad k = 0, \ldots, N-1 \label{eq:dynamics} \\
& x_0 = x(t) \label{eq:initial} \\
& |\theta_k| \leq \theta_{\max}, \quad k = 1, \ldots, N \label{eq:angle_const} \\
& |\dot{\theta}_k| \leq \dot{\theta}_{\max}, \quad k = 1, \ldots, N \label{eq:rate_const} \\
& |u_{k,i}| \leq \tau_{\max}, \quad k = 0, \ldots, N-1, \; i \in \{1,2\} \label{eq:input_const}
\end{align}
\end{subequations}


where $A \in \mathbb{R}^{6 \times 6}$ and $B \in \mathbb{R}^{6 \times 2}$ are the discrete-time linearized system matrices.
\\ \\
This MPC formulation balances the competing objectives of maintaining vertical balance (pitch regulation), tracking desired trajectories ($x, y$ position), and controlling heading (yaw) while respecting physical actuator and geometrical limits. Results were simulated in Mujoco as well as tested on the physical robot.

\subsection*{Validation Data}
\begin{table}[H]
    \centering
    \caption{Circular motion tracking (scripts/debug/debug\_velocity\_circle.py, see docs/CIRCULAR\_MOTION\_ANALYSIS.md).}
    \begin{tabular}{lccc}
        \hline
        Metric & Target & Measured & Error \\
        \hline
        Forward velocity [m/s] & 0.200 & 0.201 & 0.4\% \\
        Yaw rate [rad/s] & 0.400 & 0.057 & 86\% \\
        \hline
    \end{tabular}
\end{table}

\begin{table}[H]
    \centering
    \caption{Open-loop differential torque test (scripts/debug/test\_open\_loop\_yaw.py).}
    \begin{tabular}{lccc}
        \hline
        Quantity & Model Prediction & MuJoCo Measurement & Ratio \\
        \hline
        Rotation from $\Delta \tau = 0.10$~N$\cdot$m & $3437.7^\circ$ & $21.3^\circ$ & $161 \times$ slower \\
        \hline
    \end{tabular}
\end{table}

\begin{table}[H]
    \centering
    \caption{Yaw acceleration measurement (scripts/debug/diagnose\_yaw\_coupling.py).}
    \begin{tabular}{lccc}
        \hline
        Quantity & Model Prediction & MuJoCo Measurement & Ratio \\
        \hline
        $\ddot{\psi}$ for $\Delta \tau = 0.05$~N$\cdot$m & 60.0~rad/s$^2$ & 1.0~rad/s$^2$ & $61 \times$ slower \\
        \hline
    \end{tabular}
\end{table}

\begin{table}[H]
    \centering
    \caption{Control loop timing summary (see docs/mpc\_implementation.md).}
    \begin{tabular}{lcc}
        \hline
        Scenario & Mean solve time [ms] & Notes \\
        \hline
        Warm-started MPC & 8--12 & CasADi Opti solver, $N=20$ steps \\
        Cold-start MPC & $\sim$200 & First iteration only, subsequent warm starts recover \\
        Control loop budget & 65 & 16~Hz loop period from $T_s = 0.065$~s \\
        \hline
    \end{tabular}
\end{table}

